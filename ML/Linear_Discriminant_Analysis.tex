{\fontsize{12pt}{22pt} \textbf{Linear discriminant analysis}\par}

\vspace{5mm}

We focus on the binary case, that is when $Y=+1$ or $Y=-1$.

These two conditional laws need to be gaussians with same covariance: \vspace{1mm}

$X | Y=+1 \sim \mathcal{N}(\mu_+,\Sigma)$ with density $f_+$

$X | Y=-1 \sim \mathcal{N}(\mu_-,\Sigma)$ with density $f_-$

Let $\pi_+$, $\pi_-$ be the simple probabilities $P(Y=+1)$, $P(Y=-1)$\vspace{3mm}

$ \mathbb{P}(Y=+1|X=x) = \frac{\mathbb{P}(Y=+1, X=x)}{\mathbb{P}(X=x)}$

$ \mathbb{P}(Y=+1|X=x) = \frac{\mathbb{P}(X=x|Y=+1) \mathbb{P}(Y=+1) }{\mathbb{P}(X=x) }$

$ \mathbb{P}(Y=+1|X=x) = \frac{f_+ \pi_+}{\mathbb{P}(X=x) }$

$ \mathbb{P}(Y=+1|X=x) = \frac{f_+ \pi_+}{\mathbb{P}(X=x|Y=+1)\mathbb{P}(Y= +1) + \mathbb{P}(X=x|Y=-1)\mathbb{P}(Y= -1) }$

$\mathbb{P}(Y=+1|X=x) = \frac{f_+ \pi_+}{(f_+\pi_+ + f_-\pi_-)}$

Similarly,

$ \mathbb{P}(Y=-1|X=x) = \frac{f_- (1-\pi_+)}{\mathbb{P}(X=x) }$

$\mathbb{P}(Y=-1|X=x) = \frac{f_- (1-\pi_+)}{(f_+\pi_+ + f_-\pi_-)}$

\vspace{3mm}

The result shows us that we can express the two conditionnal probabilities in terms of conditionnal densities and "simple" probabilities ($\pi_+$, $\pi_-$).

Recall that multivariable gaussian density is: $f(x)=\frac{1}{\sqrt{2 \pi |\Sigma|}}e^{-\frac{1}{2}(x-\mu)^T\Sigma^{-1}(x-\mu)}$

In practice, $\mu_+$, $\mu_-$, $\pi_+$ and $\Sigma$ are unknown. Thus we use empiric values:

$\widehat{\pi}_+ = m/n$

$\widehat{\mu}_+ = \frac{1}{m} \Sigma \mathbbm{1}_{\{y_i=+1\}}x_i$

$\widehat{\mu}_- = \frac{1}{n-m} \Sigma \mathbbm{1}_{\{y_i=-1\}}x_i$

$\widehat{\Sigma} = \frac{1}{n-2} ((m-1) \widehat{\Sigma}_+ + (n-m-1)\widehat{\Sigma}_-)$

$\widehat{\Sigma}_+ = \frac{1}{m-1} \Sigma \mathbbm{1}_{\{y_i=+1\}}(x_i-\widehat{\mu}_+)(x_i-\widehat{\mu}_+)^T$

$\widehat{\Sigma}_- = \frac{1}{n-m-1} \Sigma \mathbbm{1}_{\{y_i=-1\}}(x_i-\widehat{\mu}_-)(x_i-\widehat{\mu}_-)^T$

 \vspace{3mm}

\underline{Classification}

We predict class = 1 when $\mathbb{P}(Y=+1 | X) > \mathbb{P}(Y=-1 | X)$

=> $\frac{\mathbb{P}(Y=+1 | X)}{\mathbb{P}(Y=-1 | X)} > 1$

=> $\log(\frac{\mathbb{P}(Y=+1 | X)}{\mathbb{P}(Y=-1 | X)}) > 0$

Using previous conditional probability expressions, we end up with the following prediction rule:

  \begin{equation}
    \begin{cases}
      1 & \text{if}\ x^T\widehat{\Sigma}^{-1}(\widehat{\mu}_+ - \widehat{\mu}_-) > \frac{1}{2}\widehat{\mu}_+^T\widehat{\Sigma}\widehat{\mu}_+ - \frac{1}{2}\widehat{\mu}_-^T\widehat{\Sigma}\widehat{\mu}_- + \log(1-m/n) - \log(m/n) \\
      -1 & \text{otherwise}
    \end{cases}
  \end{equation}

$\widehat{\mu}_+$, $\widehat{\mu}_-$, $\widehat{\pi}_+$ and $\widehat{\Sigma}$ will be computed with \textit{train data}.

 $x$ is the \textit{test data}.

\vspace{3mm}

\textit{Note}: $\widehat{\Sigma}^{-1}(\widehat{\mu}_+ - \widehat{\mu}_-)$ is the \textbf{Fisher function} (Saporta).

\lstset{language=Python}
\lstset{frame=lines}
\lstset{caption={LDA algorithm}}
\lstset{label={lst:code_direct}}
\lstset{basicstyle=\footnotesize}
\begin{lstlisting}

class LDAClassifier():
    
    def fit(self, X, y):       
        
        X_p = X[y == 1, :]
        X_m = X[y == -1, :]
        
        X_p_x1 = X_p[:,0]
        X_p_x2 = X_p[:,1]
        X_m_x1 = X_m[:,0]
        X_m_x2 = X_m[:,1]
        
        n = len(X)
        m = len(X_p)
        
        mean_p_x1 = np.mean(X_p_x1)
        mean_p_x2 = np.mean(X_p_x2)
        mean_p = np.array([mean_p_x1,mean_p_x2]) # mu_plus (estimated)
        cov_p = np.cov(np.transpose(X_p))
        

        mean_m_x1 = np.mean(X_m_x1)
        mean_m_x2 = np.mean(X_m_x2)
        mean_m = np.array([mean_m_x1,mean_m_x2]) # mu_minus (estimated)
        cov_m = np.cov(np.transpose(X_m))
        
        cov_est = (1/(n-2))*( (m-1)* cov_p + (n-m-1)* cov_m) # sigma (estimated)
        inv_cov_est = np.linalg.inv(cov_est)
        
        a1 = np.dot(np.transpose(mean_p),inv_cov_est)
        a2 = np.dot(np.transpose(mean_m),inv_cov_est)
        
        # 2nd term in inequality
        self.alpha = 0.5*(np.dot(a1,mean_p)  - 0.5*np.dot(a2,mean_m)) + np.log(1- m/n) - np.log(m/n)
        # 1st term in inequality
        self.beta =  np.dot(inv_cov_est,mean_p-mean_m)
        
        return self
    
    def predict(self, X):
        
        y_=[]
        
        for i in range(len(X)):
            X_pred = X[i]
            beta = np.dot(np.transpose(X_pred), self.beta)
            if (beta>self.alpha):
                Y_pred = 1
            else:
                Y_pred = -1
            y_.append(Y_pred)
        return np.array(y_) 

\end{lstlisting}

\vspace{5mm}