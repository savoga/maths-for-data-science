\section*{Bias-Complexity trade-off}

\label{sec:bias-complexity-trade-off}

\vspace{5mm}

$\mathcal{H}$ = hypothesis class = all the classifiers that are considered

The size of $\mathcal{H}$ can be seen as a measure of complexity.

\vspace{5mm}

We can decompose the error of an $ERM_\mathcal{H}$ (Empiric Risk Minimization algorithm):

$$L_{\mathcal{D}}(h_s) = \epsilon_{app} + \epsilon_{est}$$

- \textit{Approximation error}: $\epsilon_{app} = min_{h \in \mathcal{H}} L_{\mathcal{D}}(h)$. This is the error done by the best predictor among those considered. It is the \textit{bias} we have in choosing a specific class $\mathcal{H}$. \\

- \textit{Estimation error}: $\epsilon_{est} = L_{\mathcal{D}}(h) - \epsilon_{app}$. This is the error difference from a used predictor and the best one. The larger $\mathcal{H}$ (complexity), the more predictors we consider and thus the larger $\epsilon_{est}$ is likely to be. \\

High complexity <=> $\epsilon_{app}$ (bias) low <=> $\epsilon_{est}$ high <=> overfitting

Low complexity <=> $\epsilon_{app}$ (bias) high <=> $\epsilon_{est}$ low <=> underfitting \\

We also call this trade-off the \textbf{bias-variance trade-off} since a high complexity leads to a high variance. For more details on the variance, see last section of the \hyperref[sec:trees]{Trees} chapter.

\vspace{5mm}