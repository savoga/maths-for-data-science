{\fontsize{12pt}{22pt} \textbf{Likelihood method}\par}

\vspace{5mm}

This method consists on finding the parameter that maximizes the likelihood of an event. It is usually done when we know the type of law of a random variable (uniform, gaussian etc.) and we are looking for the parameter that maximizes the likelihood ($\approx$ probability) that an event occurs.

\vspace{5mm}

$L(\theta; x_1,...,x_n) = \prod_{i=1}^{n}f(x_i;\theta)$ which is the product of densities across all samples.

In discrete form: $L(\theta; x_1,...,x_n) = \prod_{i=1}^{n}\mathbb{P}(X = x_i; \theta)$


\vspace{5mm}

\textit{Note (wording clarification)}: $L(\theta | X) = \mathbb{P} (X | \theta)$

$\mathbb{P} (X | \theta)$: the probability of observing an event with fixed model parameters.

$L(\theta | X)$: the likelihood of the parameters taking certain values given that we observe an event.

\vspace{5mm}

Intuitively, we want to find the $\theta$ that maximizes a certain event, that is, obtaining some data $X$ (which is why we have $X | \theta$).

We often use the log in order to get rid of power coefficients appearing with the product. \\
\textit{likelihood equation}: $\frac{d}{d\theta}ln(L(x_1,...,x_n;\theta))=0$

\vspace{5mm}

\textit{Note}: in machine learning, we use likelihood maximization in unsupervised learning when we want to estimate parameters of a distribution sample (generative models).

\vspace{5mm}